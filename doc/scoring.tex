%  \documentclass[letter]{IEEEtran}
\documentclass[letter]{article}
\usepackage[utf8]{inputenc}
\usepackage[margin=1in]{geometry}
\usepackage{tikz}
\usepackage{ulem}
\usepackage{graphics}
\usepackage{sidecap}
\usepackage{wrapfig}
\usepackage[toc,page]{appendix}
\usepackage{caption}
\usepackage{amssymb}
\usepackage{amsmath}

\usepackage{hyperref}
\usepackage{parskip}

\hypersetup{
    colorlinks,%
    citecolor=black,%
    filecolor=black,%
    linkcolor=black,%
    urlcolor=black
}

\def\dashuline{\bgroup 
  \ifdim\ULdepth=\maxdimen  % Set depth based on font, if not set already
	  \settodepth\ULdepth{(j}\advance\ULdepth.4pt\fi
  \markoverwith{\kern.15em
	\vtop{\kern\ULdepth \hrule width .3em}%
	\kern.15em}\ULon}

\newcounter{foot}
\setcounter{foot}{1}

\setlength\parindent{2em}

\author{Christopher Patton}
\date{\today}
\title{Community-based scoring of metadictionary terms}
	
\begin{document}
\maketitle

\begin{abstract}
The \textit{SeaIce} metadictionary classifies terms in three categories. \textit{Vernacular} 
terms are those for which there is no community consensus. The debate may be centered around
the term's definition or the usefulness of the term itself. The owner has the oppurtunity to 
modify the defintion or term itself; therefore, Vernacular terms are subject to 
change as the debate proceeds. \textit{Canonical} terms are those whose definitions stabilize
over time and upon which the community largely agrees. Finally, \textit{deprecated} terms 
are stable terms that the community has largely deemed to not be useful. The goal of 
\textit{SeaIce} is to evolve a set of stable, canonical terms within the context of a reputation-based 
social ecosystem by promoting vernacular terms to the canon and deprecating those deemed 
not useful. I propose a scoring and classification system that automates this process 
for the online metadictionary. In this document I refer to community as the body of SeaIce users. 
\end{abstract}

%\tableofcontents

\section{Scoring}
The first step is to quantify consensus. The most obvious definition is the percentage of the 
community that finds the term and its definition useful; we'll call this score. Let $u$ be the 
number of up-votes and $d$ be the number of down-votes. If every user casts a vote, then a 
term's score is simply 
$$ S = \frac{u}{u+d} $$
We could declare a term canonical if the term stabilizes (the owner makes no modifications for some
period of time) and the score is above some threshold (for the same time period). This applies 
in the converse case, when we deprecate a term. However, and this is the central observation, we
can't assume that every user will vote on every term as the dictionary increases in size. 
We therfore need to introduce a heuristic for community consensus based on user repuation. This
makes it possible for domain-experts to promote useful terms that don't need to be verified 
by the entire community. At the sametime, it is critical that substantial dissent can prevent
a term from entering the canon. 

A user's reputation is a positive integer value that is initially seeded in the database. A user gains 
reputation by participating in the community (voting and commenting on terms) and contributing 
terms that enter the canon. Let $R_i$ be the reputation acquired by user $i$ and $R$ be the total reputation of the users 
who have voted on a particular term. Let $r_i = R_i / R$ for each user $i$ who voted on the 
term. Let $t$ be the total number of users in the community and $v$ be the number of votes cast 
such that $d = v - u$. The weight of a user's vote is based on his or her 
reputation in the following relationship: 
$$ w_i = 1 + r_i \cdot (t-v) $$
The influence of the user's reputation decreases linearily as the number of voters for the term
approaches the total number of users. However, everyone is guarenteed one vote.  
Now let $U = \{w_1, w_2, \dots w_u\}$ be the set of weighted up-votes and $D = \{v_1, v_2, \dots v_d\}$
the set of weighted down-votes. Substituting these values into the equation for $S$, we have
\begin{align*} 
S &= \frac{\sum_U{w_i}}{\sum_U{w_i} + \sum_D{v_i}} \\
  &= \frac{u + \sum_U{r^w_i} \cdot (t - v)}{u + d + \big( \sum_U{r^w_i} + \sum_D{r^v_i} \big) \cdot (t-v)} \\
  &= \cfrac{1}{1 + \cfrac{\sum_D{r^v_i} \cdot (t-v) + d}{\sum_U{r^w_i} \cdot (t-v) + u}}
\end{align*}
This equation has the property that as $v$ approaches $t$, votes become increasingly fair (equal). 
Notice as well that we don't use a user's community-wide reputation percentage; this gives every term an 
equal chance of reaching a stable, canonical state despite the proposer's reputation. At the same time, 
it allows users with high reputation to debunk bad ones immediately with a single down-vote. 

I think this is a pretty good place to start, but we may find that a linear function (in terms of $v$) 
isn't adequate in practice. For instance, it may be useful to scale down reputation faster: 
$$ w_i = 1 + \frac{r_i \cdot t}{v} $$
Howver, we'll have a better idea of how to do this once we get users and a sense of the distribution 
of reputation values. 

\section{Stabilization and classificaiton}
Once a term stabilizes, we decide whether to deprecate it or include it in the canon. I suggest two 
conditions for this property: (1) the owner hasn't modifed the term or its defnition for some 
predefined time interval and (2) the rate at which the term's score changes has dropped below some
threshold close to zero. The time interval could be two days to start. 

Now, to decide whether to promote or demote a term, we need only to decide on reasonable threshold
values. To start, I suggest a stabile term should require 75\% consenus to be promoted to the canon. If 
consnesus drops below 25\% after it stabilizes, it should be deprecated. Otherwise, the term should 
remain in the vernacular. 

Another set of rules we need to define are when a canonical or deprecated term reenters the vernacular. 
We could base this on the rate of chang eof the score. 

\section{Software components}
In Chicago, we were brainstorming about a consistency issue related to voting and the term being 
modified. The solution we discussed was to require a call-for-votes and a score reset when the 
owner modifies the term. It was remarked that this is a bit clunky; I'd like to suggest an alternative.

\subsection{Notifications}
First, we will allow uesrs to "star" and "unstar" terms that they're interested in. 
On the homepage of the SeaIce website, we could have a notification page, much like facebook, which 
provides updates on terms you own and terms your're tracking:
\begin{enumerate}
  \item User X has commented on your term A
  \item User X has updated term B 
  \item Term B has been promoted to the canon
  \item Term C has been deprecated
  \item Term C has rentered the vernacular
\end{enumerate}

Once you see the notifcation, you can respond by commenting, changing your vote, unstarring the term,
etc. Because term stabilization depends not only on the owner not editing the term but on the score's
rate of change, there is no need to reset the score on modification. 

\subsection{Browsing and search}
We have a number of distinct goals in term discovery: most stable (canonical, lots of consensus), 
most recently contributed, and most frequently discussed to name a few. In the manner of stackoverflow, 
we'll provide a listing of terms per criterion. Making these available will be very useful for  contribution. 
The default search query ranks pages first by 
relevance to the query, second by stability and classificaition, and third by consensus score. 



\end{document}
